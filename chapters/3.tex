\section{Aturan Dasar Penulisan Kode PHP}
Seperti bahasa pemograman yang lain, PHP juga memiliki aturan penulisan seperti case sensitifity (perbedaan antara huruf besar dan kecil), cara mengakhiri sebuah baris perintah, dan pengaruh penggunakan spasi dalam membuat kode program PHP. Berikut adalah aturan dasar penulisan kode PHP:
\subsection{Case Sensitivity (perbedaan huruf besar dan kecil) dalam PHP}
PHP tidak membedakan huruf besar dan kecil (case insensitive) untuk penamaan fungsi (function), nama class, maupun keyword bawaan PHP seperti echo, while, dan class. Ketiga baris berikut akan dianggap sama dalam PHP:
\begin{lstlisting}
<?php
Echo “Hello World”;
ECHO “Hello World”;
EchO “Hello World”;
?>
\end{lstlisting}
Akan tetapi, PHP membedakan huruf besar dan huruf kecil (case sensitive) untuk penamaan variabel, sehingga akan dianggap sebagai variabel yang berbeda. Sering kali error terjadi dikarenakan salah menuliskan nama variabel, yang seharusnya menggunakan huruf kecil, ditulis dengan huruf besar.
\begin{lstlisting}
<?php
$luqman="Luqman";
echo $Luqman; // Notice: Undefined variable: Luqman
?>
\end{lstlisting}
Untuk mengatasi perbedaan ini, disarankan menggunakan huruf kecil untuk seluruh kode PHP, termasuk variabel, fungsi maupun class. Jika membutuhkan nama variabel yang terdiri dari 2 kata, karakter spasi bisa digantikan dengan underscore.
\section{Embedded Script dan Non Embedded}
\subsection{Embedded Script}
  \item Berikut merupakan contoh dokumen HTML yang akan dihasilkan dengan menggunakan program/script PHP dalam embedded script
    Ditampilkan dibawah ini  :
    \lstinputlisting[firstline=1, lastline=12]{src/embedded_script.php}
Script diatas menunjukkan contoh script PHP sederhana yang disebut dengan script embedded yang di sisipkan diantara tag-tag HTML. Script tersebut digunakan apabila isi dari suatu dokumen HTML diinginkan dari hasil eksekusi suatu script PHP. jika dilihat dari source-nya dengan menggunakan view source pada web browser maka tampilannya akan berupa seperti berikut
    \lstinputlisting[firstline=14, lastline=22]{src/embedded_script.php}
Source dokumen HTML yang tampil berupa dokumen HTML yang tidak lagi dari script PHP yang berisi script PHP karena semua menjadi tag HTML, karena pada saat dieksekusi maka bukan scriptnya yang dikirim tetapi eksekusi dari script tersebut yang dikirim 
\subsection{Non Embedded}
   \item Script PHP dibawah ini merupakan script murni dari pembuatan program dengan menggunakan PHP, tag dokumen HTML yang dihasilkan untuk membuat dokumen merupakan bagian dari script PHP. di tampilkan dibawah ini:
  \lstinputlisting[firstline=24, lastline=35]{src/embedded_script.php}
dan dibawah ini merupakan source dokumen HTML dari tampilan kode diatas  
  \lstinputlisting[firstline=38, lastline=40]{src/embedded_script.php}
Jika diperhatikan dokumen HTML tersebut tidak beraturan ditampilkan. Hal tersebut tidak menjadi masalah, yang penting adalah browser web dapat menampilkannya, karena dokumen tag HTML ini murni dihasilkan dari script PHP. 