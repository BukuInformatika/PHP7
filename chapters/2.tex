\section{Pemasangan PHP di Server Web Windows}
Untuk memasang dan menggunakan PHP di lingkungan windows maka kita harus memasang terlebih dahulu software atau program tambahan yang diperlukan oleh PHP untuk dijalankan didalam sistem operasi Windows. Program atau software tambahan untuk windows tersebut adalah Redistributable Visual C++ dari microsoft. yang harus dipasang, disesuaikan dengan sistem operasi windows yang kita gunakan apakah windows 32 bit atau 64 bit. Server web yang akan digunakan untuk dapat menjalankan program PHP yang kita kembangkankan yaitu Server web PHP built-in dan Server web eksternal. 
\par
Web server inilah yang akan menerjemahkan kode PHP menjadi HTML dan mengirimnya ke web browser untuk ditampilkan. Kita harus menyewa web server agar kode PHP dapat diproses dan diakses di internet. Namun aplikasi web server ini dapat diinstall di komputer lokal, dan inilah yang akan kita install dalam tutorial kali ini. Untuk aplikasi web server, terdapat beberapa pilihan. Saat ini web server yang sering digunakan adalah Apache, Nginx, dan Microsoft IIS. Apache dan Nginx merupakan aplikasi open source dan dapat digunakan dengan gratis. Namun kali ini kita akan menjalankan PHP menggunakan Apache, karena Apache masih menjadi aplikasi web server yang paling banyak dipakai saat ini. Akan tetapi, proses instalasi web server Apache dan PHP secara terpisah akan membutuhkan waktu yang cukup lama dan juga butuh pengetahuan tentang konfigurasi Apache. Berita baiknya, terdapat banyak aplikasi yang membundel Apache+PHP. Beberapa diantaranya adalah XAMPP dan WAMP. Pada tutorial belajar PHP ini kita akan menggunakan XAMPP. Aplikasi terakhir yang kita butuhkan adalah web browser. Disini saya akan menggunakan web browser UC Browser dan Goggle Chrome.

\subsection{Cara Install XAMPP}
XAMPP adalah singkatan dari aplikasi dalam paketnya, yaitu: X (berarti cross-platform, maksudnya tersedia dalam berbagai sistem operasi), Apache Web Server, MySQL, PHP dan Perl. Dengan menginstall XAMPP, secara tidak langsung kita telah menginstall keempat aplikasi tersebut. File download dapat di download di link berikut https://www.apachefriends.org/download.html.

\section{Server web Eksternal untuk PHP}
Server web eksternal adalah server web yang digunakan untuk keperluan publikasi dalam dokumen HTML, dan bisa juga dijadikan sebagai platform untuk aplikasi Internet yang interaktif dengan menggunakan browser web sebagai klien dari aplikasinya. Penggunaan server web eksternal sejak pengembangan aplikasi sangat dianjurkan agar lingkungan pengembangan aplikasi akan sama atau mirip dengan lingkungan server web pada saat implementasi dari aplikasi yang dikembangkan. Berikut adalah beberapa server web yang paling banyak digunakan dan dapat dipilih untuk kepentingan pengembang aplikasi.web dengan menggunakan PHP:
\begin{enumerate}
\item IIS (Internet Information System)
IIS adalah server web yang disediakan oleh microsoft Pengguna bisa mengaktifkannya pada saat instalasi atau setelah sistem operasi terpasang. Server web ini hanya ada pada sistem operasi windows.

\item Apache
Apache (http://apache.org) adalah server web open source yang dapat digunakan secara bebas oleh siapa pun yang berminat. Server web ini tersedia pada sistem operasi windows dan *nix/Linux.

\item LightTPD
Server web ini tersedia pada sistem operasi windows dan *nix/Linux.
  
\item Nginx
merupakan server web alternatif yang ringan dan tersedia untuk sistem operasi windows dan *nix/Linux.
\end {enumerate}

Aplikasi PHP yang akan dihasilkan nanti dapat dijalankan disemua server web, walaupun kita membuatnya diserver web yang berbeda.