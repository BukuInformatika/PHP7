\section{E-Logbook}
E-Logbook adalah sebuah buku elektronik untuk mencatat catatan/dokumen penting secara detail setiap aktivitas yang berisi masalah-masalah yang membutuhkan tindak lanjut dari pihak yang terlibat dalam satu hari penuh. Seluruh pegawai sebaiknya membaca buku ini agar mengetahui kegiatan, kerusakan, dan target pekerjaan apa saja yang dilakukan hari sebelumnya. Ada beberapa manfaat e-logbook antara lain:
\begin{enumerate}
\item bahan bukti untuk merekap seluruh aktivitas
\item bahan pembuatan laporan kegiatan
\item alat untuk memudahkan pegawai dalam merekap kegiatan
 \end{enumerate}

Hal yang perlu di isi dalam e-logbook ini antara lain:
\begin{enumerate}
\item Hari, tanggal dan tahun
\item Nama pegawai yang dinas pada hari tersebut
\item Nama vendor yang dinas pada hari tersebut
\item Penerbangan hari ini
\item Kegiatan
\item Kerusakan
\item Target pekerjaan
\end{enumerate}

\section{OOP Pada PHP}
OOP (Object Oriented Programming) adalah suatu metode pemrograman yang berorientasi kepada objek. Tujuan dari OOP diciptakan adalah untuk mempermudah pengembangan program dengan cara mengikuti model yang telah ada di kehidupan sehari-hari. Jadi setiap bagian dari suatu permasalahan adalah objek, nah objek itu sendiri merupakan gabungan dari beberapa objek yang lebih kecil lagi. Saya ambil contoh Pesawat, Pesawat adalah sebuah objek. Pesawat itu sendiri terbentuk dari beberapa objek yang lebih kecil lagi seperti mesin, roda, baling-baling, kursi, dll. Pesawat sebagai objek yang terbentuk dari objek-objek yang lebih kecil saling berhubungan, berinteraksi, berkomunikasi dan saling mengirim pesan kepada objek-objek yang lainnya. Begitu juga dengan program, sebuah objek yang besar dibentuk dari beberapa objek yang lebih kecil, objek-objek itu saling berkomunikasi, dan saling berkirim pesan kepada objek yang lain.
Dalam tutorial belajar OOP PHP ini kita akan membahas tentang pengertian class, object, property dan method. Keempat ‘keyword’ inilah yang menjadi pondasi dasar dari Pemrograman Berbasis Objek. Selain pengertian, kita juga akan mempelajari cara penulisannya dengan PHP.
\subsection{Pengertian Class}
\subsection{Pengertian Object}
\subsection{Pengertian Property}
\subsection{Pengertian Method}