\section{Pendahuluan}
	Situs web merupakan suatu layanan yang menyajikan informasi menggunakan konsep hyperlink, yang memudahkan pengguna dalam menelusuri atau mencari informasi dari internet untuk mendapatkan informasi, dengan cukup mengklik suatu link beupa teks atau gambar, maka dari teks atau gambar tersebut akan menampilkan informasi yang detail (rinci).
Informasi yang akan disajikan dalam halaman web menggunakan konsep multimedia, informasi dapat disajikan dengan menggunakan banyak media (teks, gambar, animasi, suara (audio), dan film). Dalam suatu halaman web, informasi akan disajikan dalam kombinasi dari media-media tersebut yang disajikan dalam suatu halaman.

Situs web berupa kumpulan informasi yang disediakan secara perorangan, kelompok, atau organisasi. yang ditempatkan setidaknya pada sebuah server web yang dapat diakses melalui jaringan seperti Internet, ataupun jaringan wilayah lokal (LAN) melalui alamat Internet yang dikenali sebagai URL. Gabungan atas semua situs yang dapat diakses publik di Internet disebut pula sebagai World Wide Web atau lebih dikenal dengan singkatan WWW. Web merupakan hal yang sangat populer di lingkungan pengguna internet dalam mengakses dan mendapatkan informasi karena kemudahan yang diberikan kepada pengguna internet untuk melakukan penelusuran, penjelajahan dan pencarian informasi   
\section{Pengenalan}
PHP (Hypertext Preprocessor) adalah bahasa pemrograman script server-side yang didesain untuk pembuatan atau pengembangan web.
Dengan ini, PHP juga dapat digunakan sebagai bahasa pemrograman umum. PHP sendiri dikembangkan pada tahun 1995
oleh Rasmus Lerdorf, dan pada akhhirnya dikelola oleh The PHP Group. Situs resmi PHP beralamat http://www.php.net.
PHP disebut bahasa pemrograman server side karena PHP diproses pada komputer server. 
Hal ini berbeda dibandingkan dengan bahasa pemrograman client-side seperti JavaScript yang diproses pada web browser (client).
\par
Pada bulan Juni 1996, dirilis PHP/FI 2.0. Pada rilis ini interpreter PHP sudah diimplementasikan dalam program C. Dalam rilis ini disertakan juga modul-modul ekstensi yang meningkatkan kemampuan PHP/FI secara signifikan. Pada tahun 1997, sebuah perusahaan bernama Zend menulis ulang interpreter PHP menjadi lebih bersih, lebih baik, dan lebih cepat. Kemudian pada Juni 1998, perusahaan tersebut merilis interpreter baru untuk PHP dan meresmikan rilis tersebut sebagai PHP 3.0.
Pada pertengahan tahun 1999, Zend merilis interpreter PHP baru dan rilis tersebut dikenal dengan PHP 4.0. PHP 4.0 adalah versi PHP yang paling banyak dipakai pada awal abad ke-21. Versi ini banyak dipakai disebabkan kemampuannya untuk membangun aplikasi web kompleks tetapi tetap memiliki kecepatan dan stabilitas yang tinggi.
\par
Pada Juni 2004, Zend merilis PHP 5.0. Dalam versi ini, inti dari interpreter PHP mengalami perubahan besar. Versi ini juga memasukkan model pemrograman berorientasi objek ke dalam PHP untuk menjawab perkembangan bahasa pemrograman ke arah paradigma berorientasi objek. PHP juga banyak diaplikasikan untuk pembuatan program-program seperti sistem informasi  klinik, rumah sakit, akademik, keuangan, manajemen aset, manajemen bengkel dan lain-lain. Dapat dikatakan bahwa program aplikasi yang dulunya hanya dapat dikerjakan untuk desktop aplikasi, PHP sudah dapat mengerjakannya. Penerapan PHP saat ini juga banyak ditemukan pada proyek-proyek pemerintah seperti e-budgetting, e-procurement, e-goverment dan e e lainnya. Website Ubaya ini juga dibuat menggunakan PHP. PHP juga dapat dilihat sebagai pilihan lain dari ASP.NET/C#/VB.NET Microsoft, ColdFusion Macromedia, JSP/Java Sun Microsystems, dan CGI/Perl. Contoh aplikasi lain yang lebih kompleks berupa CMS yang dibangun menggunakan PHP adalah Wordpress, Mambo, Joomla, Postnuke, Xaraya, dan lain-lain.

\section{CGI, FastCGI dan Modul Server Web}
CGI(Common Gateway Interface) adalah suatu standar yang menghubungkan (interface) aplikasi internal dengan server web.  

FastCGI merupakan standar baru yang menambah dan meningkatkan kemampuan dari program CGI, dimana FastCGI merupakan standar  terbuka yang banyak digunakan oleh server web komersial dan opensource.

CGI akan di eksekusi setiap  kali program CGI dipanggil, seteleh dieksekusi CGI tersebut langsung dihapus dari memori. FastCGI akan terus berada di dalam memori, sampai ada perintah  yang secara eksplisit untuk menghentikan program tersebut, sehingga FastCGI akan lebih cepat untuk melayani suatu permintaan. Keunggulan dari FastCGI adalah FastCGI  berada pada komputer yang berbeda dengan server webnya.

Modul server web merupakan standar yang mengintegrasikan aplikasi luar menjadi bagian dari server web. PHP merupakan salah satu program yang merupakan bagian dari server web. Tidak semua program CGI dibuatkan menjadi modul untuk server web, dan tidak semua server web menyediakan slot untuk ditambahi modul server web dari luar. 
