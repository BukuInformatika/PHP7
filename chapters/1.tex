\section{Pendahuluan}
	Situs web merupakan suatu layanan yang menyajikan informasi menggunakan konsep hyperlink, yang memudahkan pengguna dalam menelusuri atau mencari informasi dari internet untuk mendapatkan informasi, dengan cukup mengklik suatu link beupa teks atau gambar, maka dari teks atau gambar tersebut akan menampilkan informasi yang detail (rinci).
Informasi yang akan disajikan dalam halaman web menggunakan konsep multimedia, informasi dapat disajikan dengan menggunakan banyak media (teks, gambar, animasi, suara (audio), dan film). Dalam suatu halaman web, informasi akan disajikan dalam kombinasi dari media-media tersebut yang disajikan dalam suatu halaman.
	Situs web berupa kumpulan informasi yang disediakan secara perorangan, kelompok, atau organisasi. yang ditempatkan setidaknya pada sebuah server web yang dapat diakses melalui jaringan seperti Internet, ataupun jaringan wilayah lokal (LAN) melalui alamat Internet yang dikenali sebagai URL. Gabungan atas semua situs yang dapat diakses publik di Internet disebut pula sebagai World Wide Web atau lebih dikenal dengan singkatan WWW. Web merupakan hal yang sangat populer di lingkungan pengguna internet dalam mengakses dan mendapatkan informasi karena kemudahan yang diberikan kepada pengguna internet untuk melakukan penelusuran, penjelajahan dan pencarian informasi   
\section{Pengenalan}
PHP adalah bahasa pemrograman script server-side yang didesain untuk pembuatan atau pengembangan web.
Dengan ini, PHP juga dapat digunakan sebagai bahasa pemrograman umum. PHP sendiri dikembangkan pada tahun 1995
oleh Rasmus Lerdorf, dan pada akhhirnya dikelola oleh The PHP Group. Situs resmi PHP beralamat http://www.php.net.
PHP disebut bahasa pemrograman server side karena PHP diproses pada komputer server. 
Hal ini berbeda dibandingkan dengan bahasa pemrograman client-side seperti JavaScript yang diproses pada web browser (client).


