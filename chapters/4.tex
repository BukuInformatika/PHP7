\section{Array}
Array adalah variabel jamak, yang mempunyai banyak elemen yang diacu dengan satu nama yang sama. Array (atau larik dalam bahasa indonesia) bukanlah tipe data dasar seperti integer atau boolen, Array adalah sebuah tipe data bentukan yang terdiri dari kumpulan tipe data lainnya. Menggunakan array akan memudahkan dalam membuat kelompok data, serta menghemat penulisan dan penggunaan variabel. berikut sebagai contoh
 \begin{lstlisting}
<?php
    $a = array("budi", 20, 58.5);
?>
\end{lstlisting}
Array dalam PHP juga merupakan tipe data, bukan sekedar variabel. Berikut merupakan jenis array dalam PHP:
\subsection{Array Berindeks}
Array berindeks adalah array yang diindeks berdasarkan nomor/angka. Indeks array pada umumnya dimulai dari angka 0. Anda bebas mendefinisikan indeks dengan nilai yang Anda tentukan.

\begin{table}[h]
\caption(Array Berindeks}
\centering
\begin{tabular}{|c|c|c|c|c|}
\hline
\textbf{10}&\textbf{20}&\textbf{30}&\textbf{40}&\textbf{50}\\
\hline

\begin{lstlisting}
 $a[0]
\end{lstlisting}  & 

\begin{lstlisting}
 $a[1]
\end{lstlisting} &

\begin{lstlisting}
 $a[2]
\end{lstlisting} &

\begin{lstlisting}
 $a[3]
\end{lstlisting} &

\begin{lstlisting}
 $a[4]
\end{lstlisting}\\
\hline
\end{tabular}
\label{tabel : Array Berindeks}
\end{table}


Contoh diatas menunjukan array dengan 5 buah elemen. Elemen pertama ($a[0]) bernilai 10, elemen
kedua ($a[1]) bernilai 20, dan seterusnya. Dalam array berindeks, antara kunci (indeks) dan nilai tidak
memiliki keterkaitan.

\subsection{Array Asosiatif}
Array asosiatif adalah array yang diindeks berdasarkan nama tertentu. Letak perbedaan antara array berindeks dan array asosiatif adalah hanya terletak pada penamaan indeksnya saja.
\begin{table}[h]
\caption(Array Asosiatif}
\centering
\begin{tabular}{|c|c|c|c|c|}
\hline
\textbf{10}&\textbf{20}&\textbf{30}&\textbf{40}&\textbf{50}\\
\hline

\begin{lstlisting}
 $a["satu"]
\end{lstlisting}  & 

\begin{lstlisting}
 $a["dua"]
\end{lstlisting} &

\begin{lstlisting}
 $a["tiga"]
\end{lstlisting} &

\begin{lstlisting}
 $a["empat"]
\end{lstlisting} &

\begin{lstlisting}
 $a["lima"]
\end{lstlisting}\\
\hline
\end{tabular}
\label{tabel : Array Asosiatif}
\end{table}

Array diindeks berdasarkan nama, bukan berdasarkan nomor. Pada contoh diatas indeks array bertipe string. Pada umumnya array asosiatif digunakan untuk merepresentasikan sesuatu yang kunci dan nilainya memiliki keterkaitan, misalnya sebagai berikut.

\begin{lstlisting}
 <?php
  $kota = array("jkt" => "jakarta", "bdg" => "bandung", "sby" => "surabaya");
 ?>
\end{lstlisting}  

\subsection{File & Direktori}
Dalam management file dan direktori, PHP menyediakan lebih 70 fungsi. Beberapa fungsi utama yang berhubungan dengan management file (create, write, append, dan delete), antara lain : Membuka dan membuat file.
\begin{lstlisting}
fopen ($namafile, $mode);
\end{lstlisting}
Keterangan :
\$namafile merupakan nama file yang akan dibuat, sedangkan \$mode merupakan mode akses file. Contoh:
\begin{lstlisting}
<?php
$namafile = "data.txt";
$handle = fopen ($namafile, "mode");
if (!handle) {
	echo "<b>File yang ada buat belum ada</b>";
} else {
	echo "<b>File telah berhasil dibuka</b>";
}
fclose($handle);
?>
\end{lstlisting}

\subsection{Pemrograman Berorientasi Objek Dalam PHP }
PHP pada awalnya hanya sekumpulan script sederhana. Dalam perkembangannya, dapat ditambahkan berbagai fitur pemrograman berorientasi objek. Hal ini dimulai sejak PHP 4. Dengan lahirnya PHP 5, fitur-fitur pemrograman berorientasi objek semakin mantap dan semakin cepat. Dengan PHP 7, script yang menggunakan konsep object-oriented akan lebih cepat dan lebih efisien.
\par
Pemrograman berorientasi objek atau object-oriented programming (OOP) merupakan suatu inovasi dalam pemrograman yang menggunakan object dan class. Saat ini konsep OOP sudah semakin berkembang. Hampir setiap perguruan tinggi di dunia mengajarkan konsep OOP ini pada mahasiswanya. Pemrograman yang banyak dipakai dalam penerapan konsep OOP adalah Java dan C++. OOP bukanlah sekedar cara penulisan sintaks program yang berbeda, namun dapat lebih dari itu, OOP merupakan cara pandang dalam menganalisa sistem dan permasalahan pemrograman. Dalam OOP, setiap bagian dari program adalah object. Sebuah object mewakili suatu bagian program yang akan diselesaikan. Beberapa konsep OOP dasar, antara lain :
\begin{enumerate}
\item Encapsulation (Class dan Object)
\item Inheritance (Penurunan sifat)
\item Polymorphisme
\end{enumerate}
\par
PHP khususnya PHP 7 sudah mendukung beberapa konsep OOP. Akan tetapi PHP 7 tidak mendukung konsep Multiple-inheritance dan polymorphisme.

