\section{Aplikasi E-Logbook}
E-Logbook adalah sebuah buku elektronik untuk mencatat catatan/dokumen penting secara detail setiap aktivitas yang berisi masalah-masalah yang membutuhkan tindak lanjut dari pihak yang terlibat dalam satu hari penuh. Seluruh pegawai sebaiknya membaca buku ini agar mengetahui kegiatan, kerusakan, dan target pekerjaan apa saja yang dilakukan hari sebelumnya. Ada beberapa manfaat e-logbook antara lain:
\begin{enumerate}
\item bahan bukti untuk merekap seluruh aktivitas
\item bahan pembuatan laporan kegiatan
\item alat untuk memudahkan pegawai dalam merekap kegiatan
 \end{enumerate}

Hal yang perlu di isi dalam e-logbook ini antara lain:
\begin{enumerate}
\item Hari, tanggal dan tahun
\item Nama pegawai yang dinas pada hari tersebut
\item Nama vendor yang dinas pada hari tersebut
\item Penerbangan hari ini
\item Kegiatan
\item Kerusakan
\item Target pekerjaan
\end{enumerate}

\subsection{Merancang Database}
Untuk membuat sebuah aplikasi secara sederhana, langkah pertama yang dilakukan adalah merancang tabel-tabel database. Secara sederhana, struktur database dapat dilihat sebagai berikut:
\begin{table}[h]
\caption{Tabel Logbook }
\centering
\begin{tabular}{ |p{2cm}|p{1cm}|p{1cm}|p{5cm}| }
 \hline
 \textbf{Nama Field} & \textbf{Tipe} & \textbf{Panjang} & \textbf{Keterangan}\\
 \hline
id\_logbook & int & 10 & primary key dan auto increment\\
 \hline
tanggal & date & - & tanggal pada logbook\\
 \hline
petugas & varchar & 100 & nama petugas yang dinas pada hari tersebut\\
 \hline
vendor & varchar & 100 & nama vendor yang dinas pada hari tersebut\\
 \hline
penerbangan & varchar & 100 & daftar penerbangan pada hari tersebut\\
 \hline
kegiatan & varchar & 500 & full kegiatan pegawai padasatu hari\\
 \hline
point_kerusakan & varchar & 200 & daftar alat yang rusak sampai hari tersebut\\
 \hline
target_pekerjaan & varchar & 200 & target pekerjaan alat yang rusak sampai hari tersebut\\
 \hline
id\_user & tinyint & 1 & id user admin\\
 \hline
\end{tabular}
\end{table}
Tabel logbook digunakan untuk mencatat seluruh kegiatan pegawai dan beberapa hal penting lainnya.
 \begin{table}[h]
\caption{Tabel Pegawai}
\centering
\begin{tabular}{ |p{2cm}|p{1cm}|p{1cm}|p{5cm}| }
 \hline
 \textbf{Nama Field} & \textbf{Tipe} & \textbf{Panjang} & \textbf{Keterangan}\\
 \hline
id\_pegawai & int & 10 & primary key dan auto increment\\
 \hline
pegawai & varchar & 100 & daftar nama pegawai\\
 \hline
\end{tabular}
\end{table}
Tabel pegawai digunakan untuk mendata daftar pegawai.
 \begin{table}[h]
\caption{Tabel User}
\centering
\begin{tabular}{ |p{2cm}|p{1cm}|p{1cm}|p{5cm}| }
 \hline
 \textbf{Nama Field} & \textbf{Tipe} & \textbf{Panjang} & \textbf{Keterangan}\\
 \hline
id\_user & tinyint & 2 & primary key dan auto increment\\
 \hline
username & varchar & 30 & username user\\
 \hline
password & varchar & 35 & password user\\
 \hline
nama & varchar & 50 & nama user\\
 \hline
alamat & varchar & 150 & alamat user\\
 \hline
hp & varchar & 20 & nomor handphone user\\
 \hline
level & tinyint & 1 & level user\\
 \hline
\end{tabular}
\end{table}
Tabel user adalah tabel yang mendata admin untuk masuk ke sistem.
Contoh perintah yang digunakan untuk membuat tabel logbook, yaitu:
\begin{lstlisting}
CREATE TABLE IF NOT EXISTS `logbook` (
  `id_logbook` int(10) NOT NULL,
  `tanggal` date NOT NULL,
  `petugas` varchar(100) NOT NULL,
  `vendor` varchar(100) NOT NULL,
  `penerbangan` varchar(100) NOT NULL,
  `kegiatan` varchar(500) NOT NULL,
  `point_kerusakan` varchar(200) NOT NULL,
  `target_pekerjaan` varchar(200) NOT NULL,
  `id_user` tinyint(1) NOT NULL
) ENGINE=InnoDB AUTO_INCREMENT=4 DEFAULT CHARSET=latin1;
\end{lstlisting}
Contoh perintah yang digunakan untuk tabel pegawai, yaitu:
\begin{lstlisting}
CREATE TABLE IF NOT EXISTS `pegawai` (
  `id_pegawai` int(10) NOT NULL,
  `pegawai` varchar(100) NOT NULL
) ENGINE=InnoDB AUTO_INCREMENT=16 DEFAULT CHARSET=latin1;
\end{lstlisting}
Contoh perintah yang digunakan untuk tabel user, yaitu:
\begin{lstlisting}
CREATE TABLE IF NOT EXISTS `user` (
  `id_user` tinyint(2) NOT NULL,
  `username` varchar(30) NOT NULL,
  `password` varchar(35) NOT NULL,
  `nama` varchar(50) NOT NULL,
  `alamat` varchar(150) NOT NULL,
  `hp` varchar(20) NOT NULL,
  `level` tinyint(1) NOT NULL
) ENGINE=InnoDB AUTO_INCREMENT=14 DEFAULT CHARSET=latin1;
\end{lstlisting}

\subsection{Membuat Koneksi PHP Dengan Database}
Setelah membuat database, selanjutnya baru kita membuat koneksi antara PHP dan database. Caranya buatlah sebuah file PHP dengan nama terserah kita akan tetapi disini dinamakan \textbf{koneksi.php}. Simpan file PHP tersebut di direktori localhost.

\begin{lstlisting}
<?php
    $host = "localhost";
    $username = "root";
    $password = "";
    $database = "it_division";
    $koneksi = mysqli_connect($host, $username, $password, $database);

    if (! $koneksi) {
        die("Koneksi database gagal: " . mysqli_connect_error());
    }
?>
\end{lstlisting}

\subsection{Membuat Halaman Login}
Setelah membuat file koneksi, selanjutnya kita membuat halaman login yang digunakan dalam memverifikasi user yang akan masuk ke dalam sistem. Caranya buatlah file dengan nama \textbf{index.php}.

\begin{lstlisting}
<?php

    //memulai session
    session_start();

    //jika ada session, maka akan diarahkan ke halaman dashboard admin
    if(isset($_SESSION['id_user'])){

        //mengarahkan ke halaman dashboard admin
        header("Location: ./admin.php");
        die();
    }

    //mengincludekan koneksi database
    include "koneksi.php";

?>

<!DOCTYPE html>
<html lang="en">
<head>
    <meta charset="utf-8">
    <meta http-equiv="X-UA-Compatible" content="IE=edge">
    <meta name="viewport" content="width=device-width, initial-scale=1">
    <meta name="description" content="">
    <meta name="author" content="">

    <title>Aplikasi Logbook</title>

    <!-- Bootstrap core CSS -->
    <link href="css/bootstrap.css" rel="stylesheet">
	<style type="text/css">
	body {
	  padding-top: 40px;
	  padding-bottom: 40px;
	  background-color: #eee;
	}

	.form-signin {
	  max-width: 330px;
	  padding: 15px;
	  margin: 0 auto;
	}
	.form-signin .form-signin-heading,
	.form-signin .checkbox {
	  margin-bottom: 10px;
	}
	.form-signin .checkbox {
	  font-weight: normal;
	}
	.form-signin .form-control {
	  position: relative;
	  height: auto;
	  -webkit-box-sizing: border-box;
		 -moz-box-sizing: border-box;
			  box-sizing: border-box;
	  padding: 10px;
	  font-size: 16px;
	}
	.form-signin .form-control:focus {
	  z-index: 2;
	}
	.form-signin input[type="text"] {
	  margin-bottom: -1px;
	  border-bottom-right-radius: 0;
	  border-bottom-left-radius: 0;
	}
	.form-signin input[type="password"] {
	  margin-bottom: 10px;
	  border-top-left-radius: 0;
	  border-top-right-radius: 0;
	}
	</style>

  </head>

  <body>

    <div class="container">
	<?php

    //apabila tombol login di klik akan menjalankan skript dibawah ini
	if( isset( $_REQUEST['login'] ) ){

        //mendeklarasikan data yang akan dimasukkan ke dalam database
		$username = $_REQUEST['username'];
		$password = $_REQUEST['password'];

        //skript query ke insert data ke dalam database
		$sql = mysqli_query($koneksi, "SELECT id_user, username, nama, level FROM user WHERE username='$username' AND password=MD5('$password')");

        //jika skript query benar maka akan membuat session
		if( $sql){
			list($id_user, $username, $nama, $level) = mysqli_fetch_array($sql);

            //membuat session
            $_SESSION['id_user'] = $id_user;
			$_SESSION['username'] = $username;
			$_SESSION['nama'] = $nama;
			$_SESSION['level'] = $level;

			header("Location: ./admin.php");
			die();
		} else {

			$_SESSION['err'] = '<strong>ERROR!</strong> Username dan Password tidak ditemukan.';
			header('Location: ./');
			die();
		}

	} else {
	?>
      <form class="form-signin" method="post" action="" role="form">
		<?php
		if(isset($_SESSION['err'])){
			$err = $_SESSION['err'];
			echo '<div class="alert alert-warning alert-message">'.$err.'</div>';
            unset($_SESSION['err']);
		}
		?>
        <h2 class="form-signin-heading">Login Admin</h2>
        <input type="text" name="username" class="form-control" placeholder="Username" required autofocus>
        <input type="password" name="password" class="form-control" placeholder="Password" required>
        <button class="btn btn-lg btn-primary btn-block" type="submit" name="login">Login</button>
      </form>
	<?php
	}
	?>
    </div> <!-- /container -->

	<!-- Bootstrap core JavaScript, Placed at the end of the document so the pages load faster -->
    <script src="js/jquery.min.js"></script>
    <script src="js/bootstrap.min.js"></script>

	<script type="text/javascript">
		$(".alert-message").alert().delay(3000).slideUp('slow');
	</script>
  </body>
</html>

\end{lstlisting}

\subsection{Membuat Halaman Utama}
Setelah membuat halaman login, kita membuat halaman utama yang berisi menu-menu apa saja yang ada di sistem. Buatlah file dengan nama \textbf{admin.php}. 
\begin{lstlisting}
<?php
session_start();
if( empty( $_SESSION['id_user'] ) ){
	//session_destroy();
	$_SESSION['err'] = '<strong>ERROR!</strong> Anda harus login terlebih dahulu.';
	header('Location: ./');
	die();
} else {
	include "koneksi.php";
?>
<!DOCTYPE html>
<html lang="en">
  <head>
    <meta charset="utf-8">
    <meta http-equiv="X-UA-Compatible" content="IE=edge">
    <meta name="viewport" content="width=device-width, initial-scale=1">
    <meta name="description" content="">
    <meta name="author" content="">
    
    
    <title>Aplikasi Logbook</title>
    
    <!-- Bootstrap core CSS -->
    <link href="css/bootstrap.css" rel="stylesheet">
	<link href="css/jquery-ui.min.css" rel="stylesheet">

	<style type="text/css">
	body {
	  min-height: 200px;
	  padding-top: 70px;
	}
   @media print {
	   .container {
		   margin-top: -30px;
	   }
	   #tombol,		
      .noprint {
         display: none;
      }
   }
	</style>

  </head>

  <body>

    <?php include "menu.php"; ?>

    <div class="container">

	<?php
	if( isset($_REQUEST['hlm'] )){

		$hlm = $_REQUEST['hlm'];

		switch( $hlm ){
			case 'logbook':
				include "logbook.php";
				break;
			case 'laporan':
				include "laporan.php";
				break;
			case 'user':
				include "user.php";
				break;
			case 'cetak':
				include "cetak_logbook.php";
				break;
		}
	} else {
	?>
      <!-- Main component for a primary marketing message or call to action -->
      <div class="jumbotron">
        <h2>Selamat Datang di Aplikasi Logbook</h2>

        <p>Halo <strong><?php echo $_SESSION['nama']; ?></strong>, Anda login sebagai
			<strong>
			<?php
				if($_SESSION['level'] == 1){
					echo 'Admin.';
				} else {
						echo 'Petugas.';
				}
			?>
			</strong>
		</p>
      </div>
	<?php
	}
	?>
    </div> <!-- /container -->


    <!-- Bootstrap core JavaScript, Placed at the end of the document so the pages load faster -->
    <script src="js/jquery.min.js"></script>
    <script src="js/bootstrap.min.js"></script>
	<script src="js/jquery-ui.min.js"></script>

  </body>

</html>
<?php
}
?>
\end{lstlisting}

\subsection{Membuat Halaman Menu Logbook}
Menu logbook ini terdapat beberapa bagian seperti menambah catatan logbook, edit logbook, hapus loggbok, dan cetak logbook.
Buatlah file PHP dengan nama \textbf{logbook.php}.
\begin{lstlisting}
<?php

if( empty( $_SESSION['id_user'] ) ){

	$_SESSION['err'] = '<strong>ERROR!</strong> Anda harus login terlebih dahulu.';
	header('Location: ./');
	die();
} else {
	if( isset( $_REQUEST['aksi'] )){
		$aksi = $_REQUEST['aksi'];
		switch($aksi){
			case 'baru':
				include 'logbook_baru.php';
				break;
			case 'edit':
				include 'logbook_edit.php';
				break;
			case 'hapus':
				include 'logbook_hapus.php';
				break;
			case 'cetak':
				include 'cetak_logbook.php';
				break;
		}
	} else {

		echo '

			<div class="container">
				<h3 style="margin-bottom: -20px;">Data Logbook</h3>
					<a href="./admin.php?hlm=logbook&aksi=baru" class="btn btn-success btn-s pull-right">Tambah Data</a>
				<br/><hr/>

				<table class="table table-bordered">
				 <thead>
				   <tr class="info">
					 <th width="5%">No</th>
					 <th width="10%">Tanggal</th>
					 <th width="15%">Petugas</th>
					 <th width="15%">Vendor</th>
					 <th width="10%">Penerbangan Hari Ini</th>
					 <th width="20%">Kegiatan</th>
					 <th width="20%">Point Kerusakan</th>
					 <th width="20%">Target Pekerjaan</th>
					 <th width="10%">Tindakan</th>
				   </tr>
				 </thead>
				 <tbody>';

			//skrip untuk menampilkan data dari database
		 	$sql = mysqli_query($koneksi, "SELECT * FROM logbook");
		 	if(mysqli_num_rows($sql) > 0){
		 		$no = 0;

				 while($row = mysqli_fetch_array($sql)){
	 				$no++;
	 			echo '

				   <tr>
					 <td>'.$no.'</td>
					 <td>'.date("d M Y", strtotime($row['tanggal'])).'</td>
					 <td>'.$row['petugas'].'</td>
					 <td>'.$row['vendor'].'</td>
					 <td>'.$row['penerbangan'].'</td>
					 <td>'.$row['kegiatan'].'</td>
					 <td>'.$row['point_kerusakan'].'</td>
					 <td>'.$row['target_pekerjaan'].'</td>
					 
					 <td>

					<script type="text/javascript" language="JavaScript">
					  	function konfirmasi(){
						  	tanya = confirm("Anda yakin akan menghapus data ini?");
						  	if (tanya == true) return true;
						  	else return false;
						}
					</script>

					<a href="?hlm=cetak&id_logbook='.$row['id_logbook'].'" class="btn btn-info btn-s" target="_blank">Cetak Logbook</a>

					 <a href="?hlm=logbook&aksi=edit&id_logbook='.$row['id_logbook'].'" class="btn btn-warning btn-s">Edit</a>

					 <a href="?hlm=logbook&aksi=hapus&submit=yes&id_logbook='.$row['id_logbook'].'" onclick="return konfirmasi()" class="btn btn-danger btn-s">Hapus</a>

					 </td>';
				}
			} else {
				 echo '<td colspan="8"><center><p class="add">Tidak ada data untuk ditampilkan. <u><a href="?hlm=logbook&aksi=baru">Tambah data baru</a></u> </p></center></td></tr>';
			}
			echo '
			 	</tbody>
			</table>
		</div>';
	}
}
?>
\end{lstlisting}

\subsection{Membuat Logbook Baru}
Untuk membuat logbook baru buatlah file dengan nama \textbf{logbook\_baru.php}.
\begin{lstlisting}
<?php
if( empty( $_SESSION['id_user'] ) ){

	$_SESSION['err'] = '<strong>ERROR!</strong> Anda harus login terlebih dahulu.';
	header('Location: ./');
	die();
} else {

	if( isset( $_REQUEST['submit'] )){

		$tanggal = $_REQUEST['tanggal'];
		$petugas = $_REQUEST['petugas'];
		$vendor = $_REQUEST['vendor'];
		$penerbangan = $_REQUEST['penerbangan'];
		$kegiatan = $_REQUEST['kegiatan'];
		$point_kerusakan = $_REQUEST['point_kerusakan'];
		$target_pekerjaan = $_REQUEST['target_pekerjaan'];
		$id_user = $_SESSION['id_user'];

		$sql = mysqli_query($koneksi, "INSERT INTO logbook (tanggal, petugas, vendor, penerbangan, kegiatan, point_kerusakan, target_pekerjaan, id_user) VALUES(NOW(), '$petugas', '$vendor','$penerbangan', '$kegiatan', '$point_kerusakan', '$target_pekerjaan', '$id_user')");

		if($sql == true){
			header('Location: ./admin.php?hlm=logbook');
			die();
		} else {
			echo 'ERROR! Periksa penulisan querynya.';
		}
	} else {
?>
<h2>Tambah Logbook Baru</h2>
<hr>
<form method="post" action="" class="form-horizontal" role="form">
	<div>
		<div class="form-group">
	<label for="petugas" class="col-sm-2 control-label">Petugas</label>
			<select name="petugas" class="form-control" id="petugas" style="width:250px; height:50px" required>
				<option value="" disable>Petugas</option>
			<?php

				$q = mysqli_query($koneksi, "SELECT pegawai FROM pegawai");
				while(list($pegawai) = mysqli_fetch_array($q)){
					echo '<option value="'.$pegawai.'">'.$pegawai.'</option>';
				}

			?>

			</select>
		</div>
	</div>
	<div class="form-group">
		<label for="vendor" class="col-sm-2 control-label">Vendor</label>
		<div class="col-sm-4">
			<input type="text" class="form-control" id="vendor" name="vendor" placeholder="Vendor" required>
		</div>
	</div>
	<div class="form-group">
		<label for="penerbangan" class="col-sm-2 control-label">Penerbangan Hari Ini</label>
		<div class="col-sm-4">
			<input type="text" class="form-control" id="penerbangan" name="penerbangan" placeholder="Penerbangan" required>
		</div>
	</div>
	<div class="form-group">
		<label for="kegiatan" class="col-sm-2 control-label">Kegiatan</label>
		<div class="col-sm-6">
			<textarea class="form-control" name="kegiatan" rows="4" placeholder="Kegiatan" required></textarea>
		</div>
	</div>
	<div class="form-group">
		<label for="point_kerusakan" class="col-sm-2 control-label">Point Kerusakan</label>
		<div class="col-sm-6">
			<textarea class="form-control" name="point_kerusakan" rows="4" placeholder="Point Kerusakan" required></textarea>
		</div>
	</div>
	<div class="form-group">
		<label for="target_pekerjaan" class="col-sm-2 control-label">Target Pekerjaan</label>
		<div class="col-sm-6">
			<textarea class="form-control" name="target_pekerjaan" rows="4" placeholder="Target Pekerjaan" required></textarea>
		</div>
	</div>
	<div class="form-group">
		<div class="col-sm-offset-2 col-sm-10">
			<button type="submit" name="submit" class="btn btn-success">Simpan</button>
			<a href="./admin.php?hlm=logbook" class="btn btn-danger">Batal</a>
		</div>
	</div>
</form>
<?php
	}
}
?>
\end{listing}

\subsection{Mengedit Logbook}
Untuk mengedit logbook, kita membuat file dengan nama \textbf{logbook\_edit.php}.
\begin{lstlisting}
<?php
if( empty( $_SESSION['id_user'] ) ){

	$_SESSION['err'] = '<strong>ERROR!</strong> Anda harus login terlebih dahulu.';
	header('Location: ./');
	die();
} else {

	if( isset( $_REQUEST['submit'] )){

		$id_logbook = $_REQUEST['id_logbook'];
		$tanggal = $_REQUEST['tanggal'];
		$petugas = $_REQUEST['petugas'];
		$vendor = $_REQUEST['vendor'];
		$penerbangan = $_REQUEST['penerbangan'];
		$kegiatan = $_REQUEST['kegiatan'];
		$point_kerusakan = $_REQUEST['point_kerusakan'];
		$target_pekerjaan = $_REQUEST['target_pekerjaan'];
		$id_user = $_SESSION['id_user'];
		$sql = mysqli_query($koneksi, "UPDATE logbook SET tanggal=NOW(), petugas='$petugas', vendor='$vendor', penerbangan='$penerbangan', kegiatan='$kegiatan', point_kerusakan='$point_kerusakan', target_pekerjaan='$target_pekerjaan', id_user='$id_user' WHERE id_logbook='$id_logbook'");

		if($sql == true){
			header('Location: ./admin.php?hlm=logbook');
			die();
		} else {
			echo 'ERROR! Periksa penulisan querynya.';
		}
	} else {

		$id_logbook = $_REQUEST['id_logbook'];

		$sql = mysqli_query($koneksi, "SELECT * FROM logbook WHERE id_logbook='$id_logbook'");
		while($row = mysqli_fetch_array($sql)){

?>

<h2>Edit Data Logbook</h2>
<hr>
<form method="post" action="" class="form-horizontal" role="form">
	<div>
		<div class="form-group">
	<label for="petugas" class="col-sm-2 control-label">Petugas</label>
			<select name="petugas" class="form-control" id="petugas" style="width:250px; height:50px" required>
				<option value="" disable>Petugas</option>
			<?php

				$q = mysqli_query($koneksi, "SELECT pegawai FROM pegawai");
				while(list($pegawai) = mysqli_fetch_array($q)){
					echo '<option value="'.$pegawai.'">'.$pegawai.'</option>';
				}

			?>

			</select>
		</div>
	</div>
	<div class="form-group">
		<label for="vendor" class="col-sm-2 control-label">Vendor</label>
		<div class="col-sm-3">
			<input type="text" class="form-control" id="vendor" name="vendor" value="<?php echo $row['vendor']; ?>" placeholder="Vendor" required>
		</div>
	</div>
	<div class="form-group">
		<label for="penerbangan" class="col-sm-2 control-label">Penerbangan</label>
		<div class="col-sm-3">
			<input type="text" class="form-control" id="penerbangan" name="penerbangan" value="<?php echo $row['penerbangan']; ?>" placeholder="Penerbangan" required>
		</div>
	</div>
	<div class="form-group">
		<label for="kegiatan" class="col-sm-2 control-label">Kegiatan</label>
		<div class="col-sm-3">
			<input type="text" class="form-control" id="kegiatan" name="kegiatan" value="<?php echo $row['kegiatan']; ?>" placeholder="Kegiatan" required>
		</div>
	</div>
	<div class="form-group">
		<label for="point_kerusakan" class="col-sm-2 control-label">Point Kerusakan</label>
		<div class="col-sm-3">
			<input type="text" class="form-control" id="point_kerusakan" name="point_kerusakan" value="<?php echo $row['point_kerusakan']; ?>" placeholder="Point Kerusakan" required>
		</div>
	</div>
	<div class="form-group">
		<label for="target_pekerjaan" class="col-sm-2 control-label">Target Pekerjaan</label>
		<div class="col-sm-3">
			<input type="text" class="form-control" id="target_pekerjaan" name="target_pekerjaan" value="<?php echo $row['target_pekerjaan']; ?>" placeholder="Petugas" required>
		</div>
	</div>
	<div class="form-group">
		<div class="col-sm-offset-2 col-sm-10">
			<button type="submit" name="submit" class="btn btn-success">Simpan</button>
			<a href="./admin.php?hlm=logbook" class="btn btn-danger">Batal</a>
		</div>
	</div>
</form>
<?php
	}
	}
}
?>
\end{lstlisting}

\subsection{Menghapus Logbook}
Untuk menghapus logbook kita membuat file dengan nama \textbf{logbook\_hapus.php}.
\begin{lstlisting}
<?php
if( empty( $_SESSION['id_user'] ) ){

	$_SESSION['err'] = '<strong>ERROR!</strong> Anda harus login terlebih dahulu.';
	header('Location: ./');
	die();
} else {

if(isset($_REQUEST['submit'])){

    $id_logbook = $_REQUEST['id_logbook'];

    $sql = mysqli_query($koneksi, "DELETE FROM logbook WHERE id_logbook='$id_logbook'");
    if($sql == true){
        header("Location: ./admin.php?hlm=logbook");
        die();
    }
    }
}
?>
\end{lstlisting}

\subsection{Mencetak Logbook}
Untuk mencetak logbook kita dapat membuat file dengan nama \textbf{cetak\_loggbok.php}
\begin{lstlisting}
<?php
    if( empty( $_SESSION['id_user'] ) ){

    	$_SESSION['err'] = '<strong>ERROR!</strong> Anda harus login terlebih dahulu.';
    	header('Location: ./');
    	die();
    } else {
?>

<html>
<head>
    <link href="css/bootstrap.css" rel="stylesheet"/>
</head>
<body onload="window.print()">
    <div class="container">

<?php

    $id_logbook = $_REQUEST['id_logbook'];

    $sql = mysqli_query($koneksi, "SELECT tanggal, petugas, vendor, penerbangan, kegiatan, point_kerusakan, target_pekerjaan, id_user FROM logbook WHERE id_logbook='$id_logbook'");

    list($tanggal, $petugas, $vendor, $penerbangan, $kegiatan, $point_kerusakan, $target_pekerjaan, $id_user) = mysqli_fetch_array($sql);

    echo '
        <center><h3>Laporan Logbook</h3></center>
        <hr/>
        <h4>Tanggal : <b>'.$tanggal.'</b></h4>
        <table class="table table-bordered">
         <thead>
           <tr class="info">
           <th width="10%">Tanggal</th>
             <th width="15%">Petugas</th>
             <th width="12%">Vendor</th>
             <th width="10%">Penerbangan</th>
             <th width="10%">Kegiatan</th>
             <th width="10%">Point Kerusakan</th>
             <th width="10%">Target Pekerjaan</th>
           </tr>
         </thead>
         <tbody>

           <tr>
             <td>'.date("d M Y", strtotime($tanggal)).'</td>
             <td>'.$petugas.'</td>
             <td>'.$vendor.'</td>
             <td>'.$penerbangan.'</td>
             <td>'.$kegiatan.'</td>
             <td>'.$point_kerusakan.'</td>
             <td>'.$target_pekerjaan.'</td>
             <tr/>

        </tbody>
    </table>

    <div style="margin: 0 0 50px 75%;">
        <p style="margin-bottom: 60px;">Admin</p>
        <p><p>';

        
        echo "<b><u>Luqman Nurfajri</u></b>";

        echo '</p></p>
    </div>

    <center>-------------------- Terima Kasih ------------------- </center>

    </div>
</body>
</html>';
}
?>
\end{lstlisting}